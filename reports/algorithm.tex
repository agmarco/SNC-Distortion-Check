\title{An Algorithm for Measuring Geometric Distortion in MR and CT Phantom Images}
\author{
J. David Giese\\
Innolitics, LLC\\
\and
Yujan Shrestha, M.D.\\
Innolitics, LLC\\
}
\date{\today}

\documentclass[12pt]{article}


\begin{document}
\maketitle

\section*{Introduction}
MR scanners geometrically distort the objects they image.  The geometric distortion exists because the scanner's field is non-uniform, due to inescapable inhomogeneities in the static and gradient magnets as well as magnetic susceptibility differences in the sample.

Our algorithm calculates the geometric distortion introduced by an MR scanner from an image of a phantom with known internal structure.  We calculate the displacement of structural features in the phantom from their known locations---the magnitude and direction of the displacement is a direct measure of the geometric distortion.

The first section of this report is a semi-formal discussion of the problem.  It includes several assumptions we make about the phantom images.  The second section presents an overview of our algorithm.  The third section outlines a system for testing our algorithm, and includes preliminary results.  The final section discusses the limitations of the algorithm and directions for future improvements.

We note that our algorithm should work equally well on CT images, however, we focus exclusively on MR images in this report.

\section*{Problem}
Calculating the geometric distortion from the phantom images is essentially a non-rigid registration problem. Unlike many registration problems which are performed on two images, this problem involves an image and a set of known feature locations.  We only need to detect features in the one image because we know the feature locations in the actual phantom.

The result of this non-rigid registration is a transformation mapping the space in the MR scanner to locations in the actual phantom.  This transformation will contain a rigid component (e.g. translations and rotations) as well as a non-rigid component.  The rigid component of the transformation is due to mis-alignment of the phantom within the scanner, and is uninteresting.  The non-rigid component of the translation is a measurement of the geometric distortion.

Non-rigid registration is an open research area with many difficult probl

Fortunately it is a tractable problem because we can make a number of assumptions about the images we are registering.

\section*{Solution}
  We know the features locations because we can assume there is no distortion near the isocenter and we know the relative position of the features within the phantom.

\section*{Results}

\section*{Discussion}

\bibliographystyle{abbrv}
\bibliography{main}

\end{document}

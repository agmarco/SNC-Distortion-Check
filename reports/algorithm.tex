\title{An Algorithm for Measuring Geometric Distortion in MR and CT Phantom Images}
\author{
J. David Giese\\
Innolitics, LLC\\
\and
Yujan Shrestha, M.D.\\
Innolitics, LLC\\
}
\date{\today}

\documentclass[12pt]{article}


\begin{document}
\maketitle

\section*{Introduction}
MR scanners can geometrically distort the objects they image.  For many applications small distortions are not clinically relevant---for others they are relevant and worth measuring and correcting.  A detailed discussion of the causes of geometric distortion in MRIs and the applications in which they matter is beyond the scope of this report, however, there are good articles that cover these topics \cite{baldwin2007,torfeh2015,wang2005,mribook}.  

This report presents the steps we have taken towards developing a robust algorithm to measure geometric distortion introduced by MR scanners using images of grid phantoms.  Our algorithm calculates the geometric distortion from these phantom images using the displacement of structural features in the images from their known locations in the object (i.e. the grid phantom).

The first section of this report presents a formalized representation of the problem our algorithm is solving.  It includes several assumptions we make about the phantom images.  The second section presents an overview of our algorithm.  The third section outlines a system for testing our algorithm, and includes preliminary results.  The final section discusses the limitations of the algorithm and directions for future improvements.

\section*{Problem}
Calculating a scanner's geometric distortion from grid phantom images is a non-rigid registration problem \cite{hill2001}.  We are registering an image of the phantom containing geometric distortion with an image of the phantom that does not contain geometric distortion.

The registration produces a transformation which maps locations in the MR scanner to locations in the grid phantom.  This transformation will contain a rigid component (i.e. translation and rotation) and a non-rigid component.  The rigid component is assumed to originate from misalignment of the phantom within the scanner.  The non-rigid component is a measurement of the geometric distortion.

The distortionless image can be a CT of the phantom---where we assume that the CT does not introduce its own geometric distortion---or, if we are willing to neglect phantom manufacturing errors, it can be derived from knowledge of the phantom's design.

Non-rigid registration is an open research area with many unsolved problems.  Fortunately, we can make several assumptions about the images we are registering which make the problem tractable.

\section*{Solution}
  We know the features locations because we can assume there is no distortion near the isocenter and we know the relative position of the features within the phantom.

\section*{Results}

\section*{Discussion}

\bibliographystyle{ieeetr}
\bibliography{./algorithm.bib}

\end{document}
